\documentclass{article}
\usepackage{amsmath}
\usepackage{amssymb}
\usepackage{color}
\usepackage{geometry}
\usepackage{tabularx}
\usepackage{float}
\usepackage{graphicx}
\geometry{left=1.5cm}
\author{GONG,Xianjin}
\title{Homework 5 of Honor Calculus}

\begin{document}
\maketitle

\vspace{3.5mm}

\textcolor[rgb]{0.00,0.00,0.50}{Cartwright's Proof}\\

Consider the integrals:\\

$\displaystyle I_{n} (x) = \int_{-1}^1 (1-z^2)^n \cos{(xz)} dz$\\

where $n$ is a non-negative integer.

Then integrate it by parts, we can get:\\

$\displaystyle I_n (x) = \left[\frac{1}{x} \sin{(xz)} (1-z^2)^n\right]_{z = -1}^{z = 1} - \int_{-1}^1 \frac{1}{x} \sin{(xz)} \cdot n(1-z^2)^{n-1} \cdot (-2z) dz$\\

\hskip 0.93cm $\displaystyle = - \int_{-1}^1 \frac{1}{x} \sin{(xz)} \cdot n(1-z^2)^{n-1} \cdot (-2z) dz$\\

Integrate it by parts again, we can get:\\

$\displaystyle I_n(x) = \left[\frac{1}{x^2} \cos{(xz)} \cdot n(1-z^2)^{n-1} \cdot (-2z)\right]_{z = -1}^{z = 1} - \int_{-1}^1 \frac{1}{x^2} \cos{(xz)} \left[n \cdot (n-1) (1-z^2)^{n-2} (-2z)^2 + (-2) n (1-z^2)^{n-1}\right] dz$\\

\hskip 0.93cm $\displaystyle = - \int_{-1}^1 \frac{1}{x^2} \cos{(xz)} 2n(1-z^2)^{n-2} \left[(n-1) (2z^2) - (1-z^2)\right] dz$\\

\hskip 0.93cm $\displaystyle = \int_{-1}^1 \frac{1}{x^2} \cos{(xz)}2n(1-z^2)^{n-2}\left[(2n-1)(1-z^2) - 2(n-1)\right] dz$

\hskip 0.93cm $\displaystyle = \frac{1}{x^2} \cdot 2n(2n-1) \int_{-1}^1 \cos{(xz)} (1-z^2)^{n-1} dz - \frac{1}{x^2} \cdot 4n(n-1) \int_{-1}^1 \cos{(xz)}(1-z^2)^{n-2} dz$\\

\hskip 0.93cm $\displaystyle = \frac{1}{x^2} \left(2n(2n-1) I_{n-1}(x) - 4n(n-1) I_{n-2}(x)\right)$\\

$\therefore$ \qquad $\displaystyle x^2 I_n(x) = 2n(2n-1) I_{n-1}(x) -4n(n-1) I_{n-2}(x)$ \hskip 1.5cm $(n \geq 2)$\\

If\\

$\displaystyle J_n(x) = x^{2n+1} I_n(x)$\\

then this becomes\\

$\displaystyle J_n(x) = 2n(2n-1) J_{n-1}(x) -4n(n-1)I_{n-1}(x)$\\

Futhermore, we can prove $\forall n \in Z_+$\\

$\displaystyle J_n(x) = x^{2n+1} I_n(x) = n! \left(P_n(x) \sin{(x)} + Q_n(x) \cos{(x)}\right)$\\

by M.I.\\

First, we know:\\

$\displaystyle J_0(x) = x^{2 \times 0+1} \cdot I_0(x) = x \cdot \int_{-1}^1 (1-z^2)^0 \cos{(xz)} dz = \int_{-1}^1 \cos{(xz)} d(xz) = 2 \sin(x)$\\

$\displaystyle J_1(x) = x^{2 \times 1 + 1} \cdot I_1(x) = x^3 \int_{-1}^1(1-z^2)^1 \cos{(xz)} dz = x^2 \left[(1-z^2) \sin{(xz)}\right]_{z = -1}^{z = 1} - x^2 \int_{-1}^1 (-2z) \sin{(xz)} dz$\\

\hskip 0.94cm $\displaystyle = x^2 \int_{-1}^1 \sin{(xz)} \cdot 2z dz = 2 \int_{-1}^1 (xz) \sin{(xz)}d(xz) = 2\left\{\left[-xz \cos(xz)\right]_{z = -1}^{z = 1} - \int_{-1}^1 \cos{(xz)} d(xz)\right\}$\\

\hskip 0.94cm $\displaystyle = -4 \cos{x} + 4 \sin{x}$\\

Set\\ 

$\displaystyle J_n(x) = n! (P_n(x) \sin{x} + Q_n(x) \cos{x})$ \quad $\displaystyle J_{n+1}(x) = (n+1)! (P_{n+1}(x) \sin{x} + Q_{n+1}(x) \cos{x})$\\

then\\

$\displaystyle J_{n+2}(x) = (2n+4) (2n+3) J_{n+1}(x) - 4(n+2)(n+1)x^2J_{n}(x)$\\

\hskip 1.32cm $\displaystyle = (n+2)! \cdot \left((2(2n+3) P_{n+1}(x) - 4x^2P_n(x)) \sin{x} + (2(2n+3) Q_{n+1}(x) - 4x^2 Q_n(x)) \cos{x}\right)$\\

$\because$ \qquad $P_n(x)$ and $Q_n(x)$ are both polynomials of degree at most 2n and have integer codfficients\\

$\therefore$ \qquad $(2(2n+3) P_{n+1}(x) - 4x^2P_n(x))$ and $(2(2n+3) Q_{n+1}(x) - 4x^2 Q_n(x))$ are polynomials  of degree at most 2n+2 and have integer coefficients\\

$\therefore$ \qquad $J_{n+2}(x) = (n+2)! \cdot (P_{n+2}(x) \sin{(x)} + Q_n(x) \cos{(x)})$\\

$\therefore$ \qquad $\displaystyle J_n(x) = x^{2n+1} I_n(x) = n! \left(P_n(x) \sin{(x)} + Q_n(x) \cos{(x)}\right)$\\

\hskip 1.1cm where $P_n(x)$ and $Q_n(x)$ are polynomials of degree $\leq 2n$,\\

\hskip 1.1cm and with integer coefficients (depending on n)\\

Take $x = \pi/2$, and suppose if possible that $\pi/2 = a/b$, where $a$ and $b$ are natural numbers (i.e., assume that $\pi$ is rational). Then\\

$\displaystyle \frac{a^{2n+1}}{n!} I_n\left(\frac{\pi}{2}\right) = P_n\left(\frac{\pi}{2}\right) b^{2n+1}$\\

We know that\\

$\forall \epsilon > 0, \exists N = \max \left\{a^2, \frac{a \cdot a^2! \cdot \epsilon}{(a^2)^{a^2+1}}\right\} $ s.t.\\

$\displaystyle n > N \Rightarrow \frac{a^{2n+1}}{n!} = \frac{1}{a} \cdot \frac{(a^2)^n}{n!} = \frac{1}{a} \cdot \frac{a^2}{1} \cdot \frac{a^2}{2} \cdots \frac{a^2}{a^2} \cdot \frac{a^2}{a^2 +1} \cdots \frac{a^2}{n} < \frac{1}{a} \cdot \frac{(a^2)^{a^2}}{a^2!} \cdot \frac{a^2}{n} < \epsilon$\\

$\therefore$ $\frac{a^{2n+1}}{n!} \rightarrow 0$ when $n \rightarrow \infty$\\

Besides, when $z \in [-1,1]$\\

$0 < (1-z^2)^n < 1$ \quad $0 < cos(xz) < 1$\\

$\therefore$ \qquad $0 < I_n(x) < 2$ because the area of  it is bounded by a rectangle with area equal to 2.\\

$\because$ \qquad $P_n$ is a polynomial of degree $\leq 2n$, set $P_n(x) = p_0 +p_1x + \cdots + p_{2n}x^{2n}$\\

$\displaystyle P_n \left(\frac{\pi}{2}\right)b^{2n+1} = P_n \left(\frac{a}{b}\right)b^{2n+1} = p_0 b^{2n+1} + p_1 a b^{2n} + \cdots +p_{2n}a^{2n}b$\\

which is an integer\\

But when $n \to \infty$, the left hand side is less than 1 and greater than 0, which is contradictory since we can't find such an integer.\\

$\therefore$ \qquad $\pi$ is irrational.\\

\vskip 3cm

\textcolor[rgb]{0.00,0.00,0.50}{Bourbaki's Proof}\\

For each natural number $b$ and each non-negative integer $n$, define\\

$\displaystyle A_n(b) = b^n \int_0^{\pi} \frac{x^n (\pi-x)^n}{n!} \sin{(x)} dx$\\

Since $A_n(b)$ is the integral of a function which defined on $[0,\pi]$ that takes the value 0 on 0 and on $\pi$ and which is greater than 0 otherwise, $A_n(b) > 0$.\\

Besides\\

$\because$ \qquad $\displaystyle x(\pi - x) \leq \left(\frac{\pi}{2}\right)^2$\\

$\therefore$ \qquad $\displaystyle 0 \leq A_n(b) \leq \pi b^n \frac{1}{n!} \left(\frac{\pi}{2}\right)^{2n} = \pi \frac{(b\pi^2/4)^n}{n!}$\\

$\therefore$ \qquad $\forall \epsilon > 0$ \quad $\forall b \in N$ \quad $\displaystyle \exists M = \max \left\{b\pi^2/4,\frac{(b\pi^2/4)! \cdot \epsilon}{(b\pi^2/4)^{(b\pi^2/4)+1}}\right\}$ s.t.\\

$\displaystyle n > M \Rightarrow \pi \frac{(b\pi^2/4)^n}{n!} = \pi \cdot \frac{(b\pi^2/4)}{1} \cdot \frac{(b\pi^2/4)}{2} \cdots \frac{(b\pi^2/4)}{(b\pi^2/4)} \cdot \frac{(b\pi^2/4)}{(b\pi^2/4)+1} \cdots \frac{(b\pi^2/4)}{n}$\\

\hskip 3.36cm $\displaystyle \leq \pi \cdot \frac{(b\pi^2/4)^{(b\pi^2/4)}}{(b\pi^2/4)!} \cdot \frac{(b\pi^2/4)}{n} \leq \epsilon$\\

$\therefore$ \qquad $\lim \limits_{n \to \infty} \pi \frac{(b\pi^2/4)^n}{n!} = 0$\\

$\therefore$ \qquad $\lim \limits_{n \to \infty} A_n(b) = 0$\\

$\therefore$ \qquad for $n > M$, $0 < A_n(b) < 1$\\

On the other hand, if $a$ and $b$ are natural numbers such that $\pi = a/b$ and $f$ is the polynomial function from $[0,\pi]$ into $R$ defined by\\

$\displaystyle f(x) = \frac{x^n(a-bx)^n}{n!}$\\

then\\

$\displaystyle A_n(b) = \int_0^{\pi} f(x) \sin{x} dx = \int_0^{\pi} f(x) d(-\cos{x}) = [f(x)(- \cos{x})]_0^{\pi} - \int_0^{\pi} f'(x)(-\cos{x}) dx$\\

\hskip 0.97cm $\displaystyle = [f(x)(-\cos{x})]_0^{\pi} - [f'(x)(-sinx)]_0^{\pi} + \int_0^{\pi} -f^{(2)}(x) \sin{x} dx$\\

\hskip 0.97cm $\displaystyle = [f(x)(-\cos{x})]_0^{\pi} - \cdots \pm [f^{(2n)}(x) \cos{x}]_0^{\pi} \pm \int_0^{\pi} f^{(2n+1)}(x) \cos{x} dx$\\

$\because$ \qquad $f(x)$ is a polynomial function at most degree 2n\\

$\therefore$ \qquad $f^{(2n+1)}(x) = 0$\\

$\because$ \qquad $\displaystyle f(x) = \frac{x^n}{n!} \sum \limits_{i=0}^n a^{n-i} (-bx)^i C_i^n = \frac{1}{n!} \sum \limits_{i=0}^n a^{n-i} \cdot (-b)^i \cdot C_i^n \cdot x^{n+i}$\\

Set\\

$a^{n-1} (-b)^i C_i^n = Z_i$ which is an integer\\

$\therefore$ \qquad $\displaystyle f^{(k)}(0) = 0$ when $k \in [0,n)$\\

\hskip 1.1cm $\displaystyle f^{(k)}(0) = \frac{Z_i}{n!} \cdot k!$ when $k \in [n, 2n]$ which is an integer\\

$\because$ \qquad $f(\pi - x) = f(\frac{a}{b} - x) = f(x)$\\

$\therefore$ \qquad $f^{(k)}(\pi - x) = (-1)^kf^{(k)}(\pi - x) = f^{(k)}(x)$\\

$\therefore$ \qquad $f^{(k)}(\pi - x) = (-1)^k f^{(k)}(0)$ is an integer too.\\

So the right hand side is an integer but it should be less than one, which is contradictory.\\

$\therefore$ \qquad $\pi$ is irrational\\

\vskip 3cm

\textcolor[rgb]{0.00,0.00,0.50}{\#3}\\

According to the question:\\

$\displaystyle n(n+1) \int_{-1}^1 P_m(x) P_n(x) dx = \int_{-1}^1 P_m(x)[n(n+1)P_n(x)] dx = \int_{-1}^1 P_m(x)[2xP'_n(x) - (1-x^2)P^{(2)}(x)] dx$\\

\hskip 4.35cm $\displaystyle = \int_{-1}^1 2x P_m(x)P'_n(x) dx - \int_{-1}^1 (1-x^2) P_m(x) P^{(2)}(x) dx$\\

\hskip 4.35cm $\displaystyle = \int_{-1}^1 2x P_m(x) P'_n(x) dx - \int_{-1}^1 (1-x^2) P_m(x) d(P'_n(x))$\\

\hskip 4.35cm $\displaystyle = \int_{-1}^1 2x P_m(x) P'_n(x) dx - [(1-x^2) P_m(x) P'_n(x)]_{-1}^1 + \int_{-1}^1 (-2x) P_m(x) P'_n(x) dx +$\\

\hskip 4.69cm $\displaystyle \int_{-1}^1 (1-x^2) P'_m(x) P'_n(x) dx$\\

\hskip 4.35cm $\displaystyle = \int_{-1}^1 P'_m(x) P'_n(x) dx$\\

For the same reason\\

$\displaystyle m(m+1) \int_{-1}^1 P_m(x) P_n(x) dx = \int_{-1}^1 P'_m(x) P'_n(x) dx$\\

Therefore\\

$\displaystyle m(m+1) \int_{-1}^1 P_m(x) P_n(x) dx = n(n+1) \int_{-1}^1 P_m(x) P_n(x) dx$\\

However\\

$m \neq n$\\

Therefore\\

$\displaystyle \int_{-1}^1 P_m(x) P_n(x) dx = 0$\\

\vskip 3cm

\end{document}