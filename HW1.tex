\documentclass{article}
\usepackage{amsmath}
\usepackage{amssymb}
\usepackage{color}
\usepackage{geometry}
\usepackage{tabularx}
\usepackage{float}
\usepackage{graphicx}
\usepackage{caption}
\geometry{left=1.5cm}
\author{GONG,Xianjin}
\title{Homework 1 of Honor Calculus}

\begin{document}
\maketitle

\textcolor[rgb]{0.00,0.00,0.50}{\#1}\\

(a)\\

According to product-sum formula,\\

$\displaystyle\cos{mx}cos{nx}=\frac{1}{2}[cos(m+n)x+cos(m-n)x]$\\

$\therefore$\qquad$\displaystyle\left[-\frac{1}{m+n}\sin{(m+n)x}\right]'=\cos{(m+n)x}$, and $\displaystyle\left[-\frac{1}{m-n}\sin{(m-n)x}\right]'=\cos{(m-n)x}$\\

$\therefore$\qquad$\displaystyle\int_{0}^{2\pi}\cos{mx}cos{nx}dx=\frac{1}{2}\left[\int_{0}^{2\pi}\cos{(m+n)x}dx+\int_{0}^{2\pi}\cos{(m-n)x}dx\right]$\\

If $m\neq n$, then\\

\quad$\displaystyle\frac{1}{2}\left[\int_{0}^{2\pi}\cos{(m+n)x}dx+\int_{0}^{2\pi}\cos{(m-n)x}dx\right]$\\

$\displaystyle=\frac{1}{2}\left\{\left[-\frac{1}{m+n}\sin{(m+n)x}\right]_{0}^{2\pi}-\left[\frac{1}{m-n}\sin{(m-n)x}\right]_{0}^{2\pi}\right\}$\\

$\displaystyle=\frac{1}{2}[-0-0]=0$\\

If $m=n$, then\\

\quad$\displaystyle\frac{1}{2}\left[\int_{0}^{2\pi}\cos{(m+n)x}dx+\int_{0}^{2\pi}\cos{(m-n)x}dx\right]$\\

$\displaystyle=\frac{1}{2}\left[\int_{0}^{2\pi}\cos{(m+n)x}dx+\int_{0}^{2\pi}1dx\right]$\\

$\displaystyle=\frac{1}{2}[0+2\pi]=\pi$\\

\vspace{0.5cm}

(b)\\

$\because$\qquad$(\cos{x}+\cos{2x}+\cdots+\cos{2018x})^2$\\

\qquad$=(\cos^2{x}+\cos^2{2x}+\cdots+\cos^2{2018x})$\\

\quad\qquad$+2(\cos{x}\cos{2x}+\cdots+\cos{x}\cos{2018x}+\cos{2x}\cos{3x}+\cdots+\cos{2x}\cos{2018x}+\cdots+\cos{2017x}\cos{2018x})$\\

and according to (a)\\

$\therefore$\qquad$\int_{0}^{2\pi}(\cos{x}+\cdots+\cos{2018x})^2=2018\pi$\\

(c)\\

$\because$\qquad$f(x)\displaystyle\cdot g(x)=(a_1b_1\cos^2{x}+\cdots+a_Nb_N\cos^2{Nx})$\\

\qquad\qquad\qquad\qquad\quad$\displaystyle+2(\sum \limits_{i=2}^{N}a_1b_i\cos{x}\cos{ix}+\sum \limits_{i=3}^{N}a_2b_i\cos{2x}\cos{ix}+\cdots+a_{N-1}b_N\cos{(N-1)x}\cos{Nx})$\\

$\therefore$\qquad according to (a), we can get:

\qquad$\displaystyle\int_{0}^{2\pi}f(x)\cdot g(x)dx=\sum \limits_{i=1}^{N}a_ib_i\pi$\\

\hrule
\vskip 1mm
\hrule
\vskip 0.5cm

\textcolor[rgb]{0.00,0.00,0.50}{\#2}\\

(a)\\

\begin{figure}[H]
  \centering
  % Requires \usepackage{graphicx}
  \includegraphics[width=8cm]{HW-1-2(a).jpg}\\
  \caption*{2.(a)}
\end{figure}

As shown in figure.2.(a)\\

(b)\\

Explicitly, it is $\displaystyle F(x)=\frac{n}{2}+\frac{1}{2}(x-n)^2+cx$\\

(c)\\

Set the period as $T$, arbitrarily set $x_0\in[n,n+1]$ and $x_0+T\in[m,m+1]$\\

Because $F(x)$ is supposed to be periodic, we can get $F(x_0+T)-F(x_0)=0$\\

According to (b),\\

$\displaystyle\frac{m}{2}+\frac{1}{2}(x_0+T-m)^2+c(x_0+t)-\frac{n}{2}-\frac{1}{2}(x_0-n)^2-cx_0=0$\\

$\therefore$\qquad$\displaystyle\frac{m-n}{2}+cT+(n+T-m)x=0$\\

Because the equation above should hold regardless what x is\\

$\therefore$\qquad$n-m+T=0$ and $m-n+2xT=0$\\

\qquad\quad besides, $T\neq0$\\

$\therefore$\qquad$c=\displaystyle\frac{1}{2}$\\

\hrule
\vskip 1mm
\hrule
\vskip 0.5cm

\textcolor[rgb]{0.00,0.00,0.50}{\#3}\\

(a)\\

According to subtraction-product formula\\

$\displaystyle\cos{x^2}-\cos{x^3}=-2\sin{\displaystyle\frac{x^2+x^3}{2}}\sin{\displaystyle\frac{x^2-x^3}{2}}$\\

And $0<x^2+x^3<2, 1>x^2-x^3>0$ when $x\in(0,1)$\\

$\therefore$\qquad$0<\displaystyle\frac{x^2+x^3}{2}<\frac{\pi}{2}$ and $0<\displaystyle\frac{x^2-x^3}{2}<\frac{\pi}{2}$\\

$\therefore$\qquad$\displaystyle-2\sin{\displaystyle\frac{x^2+x^3}{2}}\sin{\displaystyle\frac{x^2-x^3}{2}}<0$ when $x\in(0,1)$\\

$\therefore$\qquad The graph of $\cos{x^2}$ is below the graph of $\cos{x^3}$.\\

\qquad\quad And because both of them are greater than 0 when $x\in(0,1)$\\

$\therefore$\qquad The area between $\cos{x^2}$ and x-axis is smaller then $\cos{x^3}$'s\\

$\therefore$\qquad$\displaystyle\int_{0}^{1}\cos{x^2}dx<\int_{0}^{1}\cos(x^3)dx$\\

(b)\\

\begin{figure}[H]
  \centering
  % Requires \usepackage{graphicx}
  \includegraphics[width=5cm]{HW-1-3-(b).jpg}\\
  \caption*{3.(b)}
\end{figure}

As shown in the figure.3.(b), the red line is the graph of $y=f(x)$ and the blue one is the graph of $y=f^{-1}(x)$\\

(c)\\

\begin{figure}[H]
  \centering
  % Requires \usepackage{graphicx}
  \includegraphics[width=5cm]{HW-1-3-(c).jpg}\\
  \caption*{3.(c)}
\end{figure}

As shown in the figure.3.(c), because for $y=f^(-1)(x)$, there is no graph when $x\in(0,1)$.\\

$\int_{\cos1}^{1}f^{-1}(x)dx$ can't contain all the area between the graph and y-axis, only the blue area 'B'.\\

However, $\int_{0}^{1}f(x)dx$ can cover all the area between the graph and x-axis, the red area 'A'.\\

$\therefore$\qquad$\int_{0}^{1}f(x)dx$ is larger than $\int_{\cos1}^{1}f^{-1}(x)dx$. And the difference is $\cos1$\\

\vskip 2cm
\hrule
\vskip 1mm
\hrule
\vskip 0.5cm

\textcolor[rgb]{0.00,0.00,0.50}{\#4}\\

According to the question, $\displaystyle\frac{1}{p}+\frac{1}{q}=1$\\

$\therefore$\qquad$\displaystyle p=\frac{q}{q-1}$ and $\displaystyle p-1=\frac{q}{q-1}-1=\frac{1}{q-1}$\\

Set $f(x)=x^{p-1}, g(x)=x^{q-1}$, then we know that $g(x)=f^{-1}(x)$\\

It means that $x=g(y)$ and $y=f(x)$ should have the same graph in the same coordinate system as figure.4.(a)\\

\begin{figure}[H]
  \centering
  % Requires \usepackage{graphicx}
  \includegraphics[width=5cm]{HW-1-4-(a).jpg}\\
  \caption*{4.(a)}
\end{figure}

Besides, $\displaystyle\int_{0}^{a}f(x)dx=\frac{a^p}{p}$ and $\displaystyle\int_{0}^{b}g(x)dx=\frac{b^q}{q}$\\

Sketch the graph and divide the question into three cases.\\

\textcircled{1} If $f(a)=b$ as shown in the graph\\

\begin{figure}[H]
  \centering
  % Requires \usepackage{graphicx}
  \includegraphics[width=3cm]{HW-1-4-(b).jpg}\\
  \caption*{4.(b)}
\end{figure}

$\because$\qquad$f(a)=b$

$\therefore$\qquad$a^{p-1}=b\Rightarrow a^p=a\cdot b=b^{\frac{1}{p-1}+1}=b^{\frac{p}{p-1}}=b^{q}$\\

And the yellow area 'C' which can represent for $a\cdot b$ is equal to the sum of the purple area 'A' which can represent for $\displaystyle\int_{0}^{a}f(x)dx$ and the blue area 'B' which can represent for $\displaystyle\int_{0}^{b}g(y)dy$\\

$\therefore$\qquad$\displaystyle ab=\frac{a^p}{p}+\frac{b^q}{q}$\\

\textcircled{2} If $f(a)>b$ as shown in the graph\\

\begin{figure}[H]
  \centering
  % Requires \usepackage{graphicx}
  \includegraphics[width=3cm]{HW-1-4-(c).jpg}\\
  \caption*{4.(c)}
\end{figure}

The yellow area 'C' which can represent for $a\cdot b$ is smaller than the sum of the purple area 'A' which can represent for $\displaystyle\int_{0}^{a}f(x)dx$ and the blue area 'B' which can represent for $\displaystyle\int_{0}^{b}g(y)dy$\\

$\therefore$\qquad$\displaystyle ab<\frac{a^p}{p}+\frac{b^q}{q}$\\

\textcircled{3} If $f(a)<b$ as shown in the graph\\

\begin{figure}[H]
  \centering
  % Requires \usepackage{graphicx}
  \includegraphics[width=3cm]{HW-1-4-(d).jpg}\\
  \caption*{4.(d)}
\end{figure}

The yellow area 'C' which can represent for $a\cdot b$ is also smaller than the sum of the purple area 'A' which can represent for $\displaystyle\int_{0}^{a}f(x)dx$ and the blue area 'B' which can represent for $\displaystyle\int_{0}^{b}g(y)dy$\\

$\therefore$\qquad$\displaystyle ab<\frac{a^p}{p}+\frac{b^q}{q}$\\

$\therefore$\qquad$\displaystyle ab\leq\frac{a^p}{p}+\frac{b^q}{q}$ and the equality only holds when $a^p=b^q$\\

\hrule
\vskip 1mm
\hrule
\vskip 0.5cm

\textcolor[rgb]{0.00,0.00,0.50}{\#5}\\

(a)\\

Set $f(x)=\displaystyle\frac{h}{c}x$ and $g(x)=\frac{h}{c-b}x-\frac{bh}{c-b}$\\

Then we can get:\\

The area of the triangle $=\displaystyle\int_{0}^{c}f(x)dx-\int_{b}^{c}g(x)dx=\left[\frac{h}{2c}x^2\right]_{0}^{c}-\left[\frac{h}{2(c-b)}x^2-\frac{bh}{c-b}x\right]_{b}^{c}=\frac{1}{2}hb$\\

(b)\\

When $c>b$ and $b>0$\\

We partition interval $[0,b]$ into n parts of equal length, and interval $[b,c]$ into m parts of equal length.\\

$\displaystyle[0,b]=U_{i=1}^{n}\left[\frac{i-1}{n}b,\frac{i}{n}b\right]$\qquad\qquad$\displaystyle[b,c]=U_{i=1}^{m}\left[\frac{i-1}{m}(c-b),\frac{i}{m}(c-b)\right]$\\

$\therefore$\qquad$A_n=U_{i=1}^{n}\left[\frac{i-1}{n}b,\frac{i}{n}b\right]\times\left[0,\frac{i-1}{n}\frac{bh}{c}\right]$\qquad$B_n=U_{i=1}^{n}\left[\frac{i-1}{n}b,\frac{i}{n}b\right]\times\left[0,\frac{i}{n}\frac{bh}{c}\right]$\\

\qquad$A_n'=U_{i=1}^{m}\left[\frac{i-1}{m}(c-b),\frac{i}{m}(c-b)\right]\times\left[\frac{i}{m}h,[b+\frac{i-1}{m}(c-b)]\frac{h}{c}\right]$\quad$B_n'=U_{i=1}^{m}\left[\frac{i-1}{m}(c-b),\frac{i}{m}(c-b)\right]\times\left[\frac{i-1}{m}h,[b+\frac{i}{m}(c-b)]\frac{h}{c}\right]$\\

$\therefore$\qquad$\displaystyle\mu(A_n)=\sum \limits_{i=1}^{n}\frac{b}{n}\cdot\frac{i-1}{n}\cdot\frac{bh}{c}=\frac{n-1}{2n}\cdot\frac{b^2h}{c}$\qquad$\displaystyle\mu(B_n)=\sum \limits_{i=1}^{n}\frac{b}{n}\cdot\frac{i}{n}\cdot\frac{bh}{c}=\frac{n+1}{2n}\cdot\frac{b^2h}{c}$\\

\qquad$\displaystyle\mu(A_n')=\sum \limits_{i=1}^{m}\frac{1}{m}(c-b)\cdot\left[\frac{bh}{c}+\frac{i-1}{m}\cdot\frac{(c-b)h}{c}-\frac{i}{m}h\right]=\frac{1}{m}(c-b)\cdot\left[\frac{mbh}{c}+\frac{m-1}{2}\frac{(c-b)h}{c}-\frac{m+1}{2}h\right]$\\

\qquad$\displaystyle\mu(B_n')=\sum \limits_{i=1}^{m}\frac{1}{m}(c-b)\cdot\left[\frac{bh}{c}+\frac{i}{m}\cdot\frac{(c-b)h}{c}-\frac{i-1}{m}h\right]=\frac{1}{m}(c-b)\cdot\left[\frac{mbh}{c}+\frac{m+1}{2}\frac{(c-b)h}{c}-\frac{m-1}{2}h\right]$\\

$\therefore$\qquad$\displaystyle\lim \limits_{n \to \infty}[\mu(B_n)-\mu(A_n)]=\lim \limits_{n \to \infty}\frac{b^2h}{c}\cdot\frac{1}{n}=0$ and $\lim \limits_{m \to \infty}[\mu(B_n')-\mu(A_n')]=\lim \limits_{m \to \infty}\left(\frac{(c-b)h}{c}\cdot\frac{1}{m}-\frac{1}{m}h\right)=0$\\

And besides, $\displaystyle\lim \limits_{n \to \infty}\mu(A_n)=\lim \limits_{n \to \infty}\mu(B_n)=\frac{b^2h}{2c}$\\

\qquad\qquad\qquad$\displaystyle\lim \limits_{m \to \infty}\mu(A_n')=\lim \limits_{m \to \infty}\mu(B_n')=\frac{(c-b)bh}{c}+\frac{(c-b^2)h}{2c}-\frac{(c-b)h}{2}$\\

\qquad\qquad\qquad\qquad$\displaystyle=bh-\frac{b^2h}{c}+\frac{ch}{2}-bh+\frac{b^2h}{2c}-\frac{ch}{2}+\frac{bh}{2}=\frac{bh}{2}-\frac{b^2h}{2c}$\\

$\therefore$\qquad The area $S=\displaystyle\frac{b^2h}{2c}+\frac{bh}{2}-\frac{b^2h}{2c}=\frac{1}{2}bh$\\

When $c=b$ and $b>0$\\

We only have the $[0,b]$ part. So the area equals to $\displaystyle\frac{b^2h}{2c}=\frac{b^2h}{2b}=\frac{bh}{2}$\\

When $0<c<b$ and $b>0$\\

We can divide the triangle into two within interval $[0,c]$ and $[c.b]$\\

And we partition interval $[0,c]$ into n parts of equal length, and interval $[c,b]$ into m parts of equal length.\\

Replacing 'b' with 'c', we know that the area of the triangle within $[0,c]$ is $\displaystyle\frac{ch}{2}$ from when $b=c$\\

With the same method we can find the area of the triangle within interval $[c,b]$ is $\displaystyle\frac{(b-c)h}{2}$\\

So the area of the whole area is $\frac{bh}{2}$\\

When $c<0$ and $b>0$\\

We can get the symmetry graph of the triangle about the y-axis and shift it with length b to the right.\\

Then it is the same as when $c>b$. The area is $\displaystyle\frac{bh}{2}$\\

When $b<0$\\

It is actually the same as above by flipping the graph bout y-axis.\\

Finally we conclude that the area of the triangle is $\displaystyle\frac{1}{2}\times base\times height$\\ 

\end{document}
